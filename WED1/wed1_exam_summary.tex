\documentclass[11pt,twoside,landscape]{article}
\usepackage[utf8]{inputenc}
\usepackage[T1]{fontenc}
\usepackage{graphicx}
\usepackage{grffile}
\usepackage{longtable}
\usepackage{wrapfig}
\usepackage{rotating}
\usepackage[normalem]{ulem}
\usepackage{amsmath}
\usepackage{textcomp}
\usepackage{amssymb}
\usepackage{capt-of}
\usepackage{hyperref}
\usepackage[top=2cm,bottom=2cm,right=2cm,left=2cm,landscape]{geometry}
\usepackage{multicol}
\usepackage{enumitem}
\setlist{noitemsep}
\setlength{\parindent}{0pt}
\setlength{\columnseprule}{0.2pt}
\setcounter{secnumdepth}{0}
\author{Olivier Lischer}
\date{\today}
\title{WED1 - Exam Summary}
\hypersetup{
 pdfauthor={Olivier Lischer},
 pdftitle={WED1 - Exam Summary},
 pdfkeywords={},
 pdfsubject={},
 pdfcreator={Emacs 27.2}, 
 pdflang={English}}


\begin{document}
\begin{multicols*}{3}

  \section{JS}
  \subsection{Array}
  Zwei for-each-Loops für Arrays: \textbf{For-In} iteriert über die Propery Name und \textbf{For-Of} über die Werte:

\begin{verbatim}
for (const x in arr) {...}
for (const x of arr) {...}
\end{verbatim}

  \subsection{Context}
\verb+this+ ist der aktuelle Context. Je nach Aufrufart wird ein anders Objekt referenziert. Bei \verb+object.foo()+ wird \verb+object+ referenziert. Bei \emph{unbound} Funktionen zeigt \verb+this+ auf das globale Objekt. Mittels \verb+apply()+ oder \verb+call()+ kann der Context gesetzt werden. Die vorherigen Regel werden ignoriert. Z.B. \verb+foo.call({ counter: 123});+



\subsection{Shim / Polyfill}
\textbf{A polyfill is a shim for a browser API}. Implementieren API's die noch nicht zur Verfügung stehen, fix JS Bugs in alten Browser, können entfernt werden, wenn nicht mehr notwendig.

\section{CSS}
\subsection{Selektor}

\begin{itemize}
\item \verb+#+ ID
\item \verb+.+ class
\item Typen (tags)
\item \verb+*+ Unveral
\item \verb+tag[attribute]+
\item \verb+a::before+ Pseudeo-Element
\item \verb+a:hover+ Pseudeo-Class
\item \verb+article.recent+ Compound (Verbindung)
\item \verb+.recent h1+ decendant (Nachfahren)
\item \verb+>+ child (Direktes Kind)
\item \verb|+| next sibling
\item \verb+~+ subsequent siblings (nachfolgende Geschwister)
\end{itemize}

\subsection{Attribut-Selektoren}
\begin{itemize}
\item \verb+[a=v]+: ist gleich v  
\item \verb+[a~=v]+: enthält v  
\item \verb+[a|=v]+: beginnt mit v-  
\item \verb+[a^=v]+: beginnt mit v
\item \verb+[a$=v]+: endet mit v
\item \verb+[a*=v]+: enthält min. einmal v
\end{itemize}

\subsection{Pseudoelment}
\begin{itemize}
\item \verb+::first-line+  
\item \verb+::first-letter+ 
\item \verb+::before+ (erstellt neues Element)
\item \verb+::after+ (erstellt neues Element)
\item \verb+selection+
\end{itemize}

\subsection{Pseudoklassen}
\begin{itemize}
\item dynamische Klassen: \verb+:hover+, \verb+:focus+, \verb+:visited+
\item strukturelle Klassen: \verb+:root+, \verb+:first-child+, \verb+:nth-child()+, \verb+:empty+
\end{itemize}

Startindex bei Pseudoklassen mit \verb+nth()+ ist immer 1. \verb+:not+ darf keine Pseudoelemente und keine \verb+not+ enthalten.

\section{DOM}
\subsection{Events}
Um Events auf einem Element zu registrieren geht man wie folgt vor:
\begin{verbatim}
const element = document.querySelector('#myID');
element.addEventListner('click', () => alert('1'));
\end{verbatim}

Eine Event durchläuft 3. Phasen: 
\begin{enumerate}
\item \textbf{Capture-Phase}: vom Root zum Leaf (jedes Element kann reagieren)
\item \textbf{Target-Phase}: Event wird auf dem /Ziel/ ausgelöst
\item \textbf{Bubble Phase}: Das Event durchläuft den DOM von Leaf zum Root (nicht jedes Event)
\end{enumerate}

\subsection{DOM Lifecycle}
DOM kennt drei unterschiedliche Zustände: 
\begin{enumerate}
\item \textbf{loading}: Dokument wird geladen
\item \textbf{interactive}: Dokument wurde geladen, gewisse Ressourcen fehlen noch (Bilder, Stylsheets, Frames)
\item \textbf{complete}: Das Dokument und alle Ressourcen sind geladen (wird durch das Event \verb+load+ representiert)
\end{enumerate}
\end{multicols*}

\begin{multicols*}{2}
\section{REST}
\subsection{Representational State Transfer (REST)}
Levels of REST
\begin{itemize}
\item Level 0 - The Swamp of POX
\item Level 1 - Resources
\item Level 2 - HTTP Verbs
\item Level 3 - Hypermedia Controls
\end{itemize}

\section{HTTP}
\subsection{Promise}
Wichtige Zustände:
\begin{itemize}
\item \textbf{pending}: Async-Operation aktiv
\item \textbf{fullfilled}: Async-Operation erfolgreich beendet, Wert verfügbar
\item \textbf{rejected}: Async-Operation erfolgreich beendet, Fehler-Detail verfügbar
\end{itemize}

Funktionen mit Promises \verb+.then()+, \verb+.cancel()+, \verb+await+, \verb+try-catch+.

\subsection{Fetch}
Making an POST-request using fetch
\begin{verbatim}
const fetchOptions = {
    "method": "POST",
    "header": {
        "accept": "application/json",
        "content-type": "application/json"
    },
    "body": JSON.stringify(song)
}
const response = await fetch(url, fetchOptions);
\end{verbatim}

\section{Node}
\begin{verbatim}
const server = http.createServer();
server.on('request', requestHandler);

server.listen(PORT,
        () => console.log('Node listening on Port ', PORT));
\end{verbatim}

\subsection{SOP / CORS}
\textbf{Same Origin Policy (SOP)}: Nur Seiten dürfen sich verbinden, die auch von diesem Server geservt wurde.\textbf{Cross Origin Resource Sharing (CORS)} wenn ein Server auch von anderen Origins kontaktiert wird, muss der Server im Header der Response das Feld \verb+Access-Control-Allow-Origin+ auf die Origin des Browser bzw. auf \verb+*+ setzten.

\end{multicols*}
\end{document}

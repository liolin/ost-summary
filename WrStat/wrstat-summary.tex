% Created 2024-01-04 Thu 09:46
% Intended LaTeX compiler: xelatex
\documentclass[11pt,twoside,landscape]{article}
\usepackage{graphicx}
\usepackage{longtable}
\usepackage{wrapfig}
\usepackage{rotating}
\usepackage[normalem]{ulem}
\usepackage{amsmath}
\usepackage{amssymb}
\usepackage{capt-of}
\usepackage{hyperref}
\usepackage{subcaption}
\usepackage[newfloat]{minted}
\usepackage{color}
\usepackage{listings}
\usepackage[top=2cm,bottom=2cm,right=2cm,left=2cm,landscape]{geometry}
\usepackage{multicol}
\usepackage{enumitem}
\usepackage{fancyhdr}
\usepackage{caption}
\usepackage{algorithm}
\usepackage{algpseudocode}
\usepackage{float}
\setlist[description]{itemsep=-1pt,leftmargin=2mm,topsep=0pt}
\setlist[itemize]{itemsep=-1pt,topsep=0pt}
\setlist{noitemsep}
\setlength{\parindent}{0pt}
\setlength{\columnseprule}{0.2pt}
\definecolor{mygreen}{rgb}{0,0.6,0}
\definecolor{mygray}{rgb}{0.5,0.5,0.5}
\definecolor{mymauve}{rgb}{0.58,0,0.82}
\lstset{ backgroundcolor=\color{white}, basicstyle=\footnotesize, breaklines=true, captionpos=b, commentstyle=\color{mygreen}, escapeinside={\%*}{*)},keywordstyle=\color{blue}, stringstyle=\color{mymauve},}
\usepackage{caption}
\author{Olivier Lischer}
\date{\today}
\title{WrStat}
\hypersetup{
 pdfauthor={Olivier Lischer},
 pdftitle={WrStat},
 pdfkeywords={},
 pdfsubject={},
 pdfcreator={Emacs 29.1 (Org mode 9.7-pre)}, 
 pdflang={English}}
\begin{document}

\pagestyle{fancy}
\fancyhf{}
\fancyhead[R]{WrStat-HS23}
\fancyhead[L]{Exam Summary}
\fancyfoot[CE,CO]{\leftmark}
\fancyfoot[R]{\thepage}
\fancyfoot[L]{Olivier Lischer}
\section{Woche 01 - Kombinatorik}
\label{sec:orga29c132}
\subparagraph{Zählregel} \
\label{sec:org6f67fc7}
\begin{itemize}
\item Disjunkte Vereinigung: \(|A \cup B| = |A| + |B|\)
\item Vereinigung: \(|A \cup B| = |A| + |B| - |A \cap B|\)
\item Paare = Produkt: \(|A \times B| = |A| \cdot |B|\)
\end{itemize}
\subparagraph{Reihenfolge / Permutation} \
\label{sec:org509686c}
Jeder Platz im Hörsaal wird belegt:
\begin{align*}
  P_n &= \underbrace{\#\text{\{Plätze für 1. Objekt\}}}_{n} \cdot \underbrace{\#\text{\{Plätze für 2. Objekte\}}}_{n-1} \cdot \ldots \cdot \underbrace{\#\{\text{Plätze für n. Objekte\}}}_{1} \\
  &= n \cdot (n-1) \cdot \ldots \cdot 1 = n!  \\
  &= \underbrace{\#\text{Plätze für 1. Objekt}}_{n} \cdot \underbrace{\#\text{Anordnung von n-1 Objekten}}_{P_{n-1}} \\
  &= n \cdot P_{n-1} = n \cdot (n - 1)! = n!
\end{align*}
\subparagraph{Anzahl / Auswahl Problem} \
\label{sec:orgdc4b352}
Auf wie viele Arten kann man \(k\) Plätze aus \(n\) Plätzen auswählen?
16 Studenten (\(k\)) platzieren sich auf 32 Plätzen (\(n\)).

\begin{align*}
\text{\#Auswahlprozesse} &= n \cdot (n - 1) \cdot \ldots \cdot (n - k + 1) \\
&= \frac{n!}{(n-k)!} \\
\text{\#Permutation} &= k \cdot (k - 1) \cdot \ldots \cdot 1 = k!
\end{align*}

\begin{align*}
C_k^n &= \frac{n \cdot (n - 1) \cdot \ldots \cdot (n-k+1)}{k \cdot (k - 1) \cdot \ldots \cdot 1} \\
&= \frac{n!}{k! (n-k)!} = \binom{n}{k}
\end{align*}

Binominal Koeffizient (funktioniert meist nicht gut, Taschenrechner können grosse \(n!\) nicht rechnen)
Besser so:
\[
\frac{n \cdot (n - 1) \cdot (n - 2) \cdot \ldots \cdot (n-k + 1)}{1 \cdot 2 \cdot 3 \cdot \ldots \cdot k}
\]
\subparagraph{Variation} \
\label{sec:orgee4c033}
Auch als Perlenkette bekannt
\[
\text{\#Möglichkeiten} = k[\text{Farben}]^{n[\text{Längen}]}
\]
\section{Woche 02 - Ereignisse und Wahrscheinlichkeit}
\label{sec:orgbf0d7df}
\subparagraph{Begriffe} \
\label{sec:org4623ff2}
\begin{center}
\begin{tabular}{ll}
Begriff & Model\\[0pt]
\hline
Versuchausgang, Elementarereignis & \(\omega\)\\[0pt]
alle Versuchsausgänge & \(\Omega\)\\[0pt]
Ereignis & \(A \subset \Omega\)\\[0pt]
Ereignis ist eingetreten & \(\omega \in A\)\\[0pt]
sicheres Ereignis, tritt immer ein & \(\Omega\)\\[0pt]
unmögliches Ereignis, kann nicht eintreten & \(\emptyset\)\\[0pt]
\(A\) und \(B\) tretten ein & \(A \cap B\)\\[0pt]
\(A\) oder \(B\) tretten ein & \(A \cup B\)\\[0pt]
\(A\) hat \(B\) zur Folge, wenn \(A\) dann auch \(B\) & \(A \subset B\)\\[0pt]
nicht A & \(\overline{A} = \Omega \setminus A\)\\[0pt]
\end{tabular}
\end{center}
\subparagraph{Bedingte Wahrscheinlichkeit} \
\label{sec:orgae1f97c}
Wahrscheinlichkeit, dass ein Toter ein rotes Shirt trägt (Wir untersuchen nur die Toten und schauen ob er ein Rotes Shirt trägt)
\[
P(R|T) = \frac{P(R \cap T)}{P(T)}
\]

Wahrscheinlichkeit, dass ein Redshirt umkommt (Wir untersuchen nur die Redshirts, und schauen ob er Tot ist):
\[
P(T|R) = \frac{P(R \cap T)}{P(R)}
\]


Daraus ergibt sich:
\begin{equation}
\begin{split} \label{eqn:dead-has-red}
P(R|T) = \frac{P(R \cap T)}{P(T)} \Rightarrow \\
P(R|T) \cdot P(T) = P(R \cap T)
\end{split} 
\end{equation}

\begin{equation}
\begin{split} \label{eqn:red-is-dead}
P(T|R) = \frac{P(R \cap T)}{P(R)} \Rightarrow \\
P(T|R) \cdot P(R) = P(R \cap T)
\end{split} 
\end{equation}

Aus \ref{eqn:dead-has-red} und \ref{eqn:red-is-dead} folgt der \textbf{Satz von Bayes}:
\[
P(R|T) \cdot P(T) = P(R \cap T) = P(T|R) \cdot P(R)
\]

\textbf{Satz der totalen Wahrscheinlichkeit}
\begin{align*}
&P(T \cap G) &= P(T|G) \cdot P(G) \\
+&P(T \cap B) &= P(T|B) \cdot P(B) \\
+&P(T \cap R) &= P(T|R) \cdot P(R) \\
=&P(T)
\end{align*}
\section{Woche 03 - bedingte Wahrscheinlichkeit}
\label{sec:org6d52b59}
\section{Woche 04 - Zufallsvariabeln}
\label{sec:orgbbe27b2}
\subparagraph{Erwartungswert} \
\label{sec:org520fe5e}

\begin{equation}
\begin{split} \label{eqn:erwartungs-wert}
E(X) = \sum_{i=1}^{n}g_iP(A_i) \\
\end{split} 
\end{equation}

\[
E(\text{"Gewinn"})=\text{"Gewinn bei Kopf"}\cdot P(\text{"Kopf"})
+\text{"Gewinn bei Zahl"}\cdot P(\text{"Zahl"})
\]
\subparagraph{Varianz} \
\label{sec:org4912069}
\[
var(X) = E(X^2) - E(X)^2
\]
\subparagraph{Table} \
\label{sec:orgd08a1cc}

\begin{center}
\begin{tabular}{rrlrr}
Ereignis & X & P(X = g) & P(X = g) * g & P(X = g) * g\textsuperscript{2}\\[0pt]
\hline
0 & 2 & 1/16 & 0.125 & 0.25\\[0pt]
1 & 3 & 4/16 & 0.75 & 2.25\\[0pt]
2 & 5 & 6/16 & 1.875 & 9.375\\[0pt]
3 & 7 & 4/16 & 1.75 & 12.25\\[0pt]
4 & 11 & 1/16 & 0.6875 & 7.5625\\[0pt]
\hline
 &  &  & E(X) & E(X\textsuperscript{2})\\[0pt]
 &  &  & 5.1875 & 31.6875\\[0pt]
 &  &  &  & var(X)\\[0pt]
 &  &  &  & 4.7773438\\[0pt]
\end{tabular}
\end{center}
\section{Woche 05 - Anwendungen der Varianz}
\label{sec:org17a30e3}
\subparagraph{Table Lineare Regression} \
\label{sec:org542694b}

\begin{center}
\begin{tabular}{rlllll}
i & x\textsubscript{i} & y\textsubscript{i} & x\textsuperscript{2}\textsubscript{i} & y\textsuperscript{2}\textsubscript{i} & x\textsubscript{iy}\textsubscript{i}\\[0pt]
\hline
1 &  &  &  &  & \\[0pt]
2 &  &  &  &  & \\[0pt]
3 &  &  &  &  & \\[0pt]
4 &  &  &  &  & \\[0pt]
5 &  &  &  &  & \\[0pt]
\hline
E &  &  &  &  & \\[0pt]
\end{tabular}
\end{center}

\[var(X) = E(X^2) - E(X)^2\]
\[var(Y) = E(Y^2) - E(Y)^2\]
\[cov(X,Y) = E(XY) - E(X)E(Y)\]
\[a = \frac{cov(X,Y)}{var(X)}\]
\[b = E(Y) - aE(X)\]
\[r = \frac{cov(X,Y)}{\sqrt{var(X)var(Y)}}\]
\section{Woche 06 - Verteilungsfunktion und Wahrscheinlichkeitsdichte}
\label{sec:org067fe3d}
Die Verteilungsfunktion beschreibt die Wahrscheinlichkeiten der Werte einer Zufallsvariable:
\[
F(X) = P(X \le x)
\]

\(\phi(x)\) ist die Ableitung von \(F(x)\) und entspricht der Verteilungsdichte Funktion.
\[
\phi(x) = \frac{d}{dx}F(x) = F'(x)
\]

\begin{itemize}
\item Wahrscheinlichkeit: \(P(X = x) \rightarrow \phi(x) dx\)
\item Summe: \(\sum_x \rightarrow \int_\infty^\infty\)
\item \(E(X) = \sum_x x \cdot P(X = x) \rightarrow E(X) = \int_\infty^\infty x \cdot \phi(x) dx\)
\end{itemize}
Wichtig: Erkennen, was ist der Wert, was ist der Erwartungswert


\begin{equation}
  E(X^2) = \int_{\infty}^{\infty} X^2 \phi(x)dx = \int_0^1 x^2 1 dx = \left[\frac{x^3}{3} \right]_0^1 = \frac{1}{3}
\end{equation}

\begin{equation}
  var(x) = \frac{1}{3} - (\frac{1}{2})^2 = \frac{4-3}{12} = \frac{1}{12}
\end{equation}

\begin{equation}
\sqrt{var(X)} = \sqrt{\frac{1}{12}} = 0.288
\end{equation}
\section{Woche 07 - Exponential- / Erlang- / Poisson-Verteilung}
\label{sec:org5a0fd29}
\subparagraph{Exponentialverteilung} \
\label{sec:org6749f2a}
\begin{description}
\item[{Dichtefunktion}] \(ae^{-ax}, a > 0\)
\item[{Verteilungsfunktion}] \(1 - e^{-ax}\)
\item[{Erwartungswert}] \(\frac{1}{a}\)
\item[{Varianz}] \(\frac{1}{a^2}\)
\item[{Median}] \(\frac{1}{a}\log{2}\)
\end{description}
\subparagraph{Possonverteilung} \
\label{sec:org91b44d3}
\begin{description}
\item[{Wahrscheinlichkeit}] \(P_\lambda(k) = \frac{\lambda^k}{k!}e^{-\lambda}\)
\item[{Erwartungswert}] \(\lambda\)
\item[{Varianz}] \(\lambda\)
\end{description}
\section{Woche 08 - Normalverteilung}
\label{sec:orgb39dd1b}
\subparagraph{Normalverteilung} \
\label{sec:org502083c}
\begin{description}
\item[{Dichtefunktion}] \(\frac{1}{\sqrt{2\pi}\sigma}e^{-\frac{(x-\mu)^2}{2\sigma^2}}\)
\item[{Verteilungsfunktion}] keine elementare Funktion (Tabelle nutzen)
\item[{Erwartungswert}] \(\mu\)
\item[{Varianz}] \(\sigma\)
\item[{Median}] \(\mu\)
\end{description}
\section{Woche 09 - Binominalverteilung}
\label{sec:org23fb94c}
\subparagraph{Binominalverteilung} \
\label{sec:orgd1f8f1e}
\begin{description}
\item[{Wahrscheinlichkeit}] \(P(k) = \binom{n}{k}p^k(1-p)^{n-k}\)
\item[{Verteilungsfunktion}] \(F(k) = \sum_{i=0}^{k}\binom{n}{i}p^i(1-p)^{n-i}\)
\item[{Erwartungswert}] \(np\)
\item[{Varianz}] \(np(1-p)\)
\end{description}
\section{Woche 10 - Schätzen}
\label{sec:org279adbe}
\subparagraph{Schätzen} \
\label{sec:org0c0e32d}
Mittelwert ist häufig ein guter Schätzer
\subparagraph{t-Verteilung} \
\label{sec:orgf34c24f}
t-Verteilung sollte dann verwendet werden, wenn man wenig Daten hat, aber es normall Verteilt ist (kleine \(n\)).
\section{Woche 11 - Hypothesentest}
\label{sec:orgffb32b5}
\subparagraph{Vorgehen Hypothesentest} \
\label{sec:org38d9832}
\begin{enumerate}
\item Nullhypothese \(H_0\) und Alternativhypothese \(H_1\)
\item Testgrösse und Verteilung unter der Annahme der Nullhypothese
\item Wahl des Signifikanzniveaus \(\alpha\)
\item Kritischer Wert für Testgrösse, die nur mit Wahrscheinlichkeit \(\alpha\) erreicht wird
\item Kritischer Wert erreicht \(\Rightarrow\) Nullhypothese \(H_0\) verwerfen
\end{enumerate}
\subparagraph{t-Test} \
\label{sec:org15cbada}
Ist der neue Dünger besser?
Die Stichproben \(X_1, \ldots, X_n\) und \(Y_1, \ldots, Y_m\) mit gleicher Varianz haben den gleichen Erwartungswert.

\[
\overline{X} = \frac{X_1 + \cdots + X_n}{n}
\]

\[
S_X^2 = \frac{1}{n-1}\sum_{i=1}^n(X_i-\overline{X})^2
\]

\[
t = \frac{\overline{X} - \overline{Y}}{\sqrt{(n-1)S_X^2 + (m-1)S_Y^2}}\sqrt{\frac{nm(n+m-2)}{n+m}}
\]

\(t_{krit}\) kann aus der t-Verteilung abgelesen werden.
\(k\) erhält man durch \(n+m-2\).
Wenn \(t_{krit}\) überschritten wird, muss \(H_0\) verworfen werden.
\section{Woche 12 - Test einer Verteilung}
\label{sec:org3e9e0b4}
\subparagraph{X\textsuperscript{2}-Test} \
\label{sec:org4a2b39e}

Table

\[
D = \sum_{i=1}^d\frac{(n_i-np_i)^2}{np_i}
\]
\subparagraph{Kolmogorov-Smirnov-Test} \
\label{sec:orgcf8b687}
\section{Woche 13 - Das Filter-Problem}
\label{sec:org966ac69}
\end{document}
\documentclass[11pt,twoside,landscape]{article}
\usepackage[top=2cm,bottom=2cm,right=2cm,left=2cm,landscape]{geometry}
\usepackage{multicol}
\usepackage{enumitem}
\setlist{noitemsep}
\setlength{\parindent}{0pt}
\setlength{\columnseprule}{0.2pt}
\usepackage[utf8]{inputenc}
\usepackage[T1]{fontenc}
\usepackage{graphicx}
\usepackage{grffile}
\usepackage{longtable}
\usepackage{wrapfig}
\usepackage{rotating}
\usepackage[normalem]{ulem}
\usepackage{amsmath}
\usepackage{textcomp}
\usepackage{amssymb}
\usepackage{capt-of}
\usepackage{hyperref}
\setcounter{secnumdepth}{0}
\author{Olivier Lischer}
\date{\today}
\title{Betriebsysteme 2}
\hypersetup{
  pdfauthor={Olivier Lischer},
  pdftitle={Betriebsysteme 2},
  pdfkeywords={},
  pdfsubject={},
  pdfcreator={Emacs 27.2 (Org mode 9.4.6)}, 
  pdflang={English}}

\begin{document}
\maketitle
\begin{multicols}{3}

\section*{Shared Object naming schema}
\begin{itemize}
\item linker lame: lib + libname + .so (libmath.so)
\item SO name: linker name + . + version number (libmath.so.2)
\item Real name: SO name + . + sub version number (libmath.so.2.1)
\end{itemize}

\section*{Amdahls Rule}
\begin{itemize}
\item T: Runtime for serially only
\item n: Number of processors
\item T': Runtime when max. parallelized
\item \(T_s\): Runtime for part which has to be executed in serial
\end{itemize}
\begin{equation*}
  s = \frac{T_s}{T}
\end{equation*}
\begin{equation*}
  f \leq \frac{1}{s + \frac{1-s}{n}}
\end{equation*}

\section*{Semaphore / Mutex}
\begin{itemize}
\item sem\_post
\item sem\_wait
\end{itemize}

\section*{Unicode}
\subsection*{UTF-8}
\begin{itemize}
\item 2 bytes
  \begin{itemize}
  \item \(U_1 = B_1 = \mathbf{110}P_10 \dots P_6 (+\mathbf{C0_h})\)
  \item \(U_0 = B_0 = \mathbf{10}P_5 \dots P_0 (+\mathbf{80_h})\)
  \end{itemize}

\item 3 bytes
  \begin{itemize}
  \item \(U_2 = B_2 = \mathbf{1110}P_15 \dots P_12 (+\mathbf{E0_h})\)
  \item \(U_1 = B_1 = \mathbf{10}P_11 \dots P_6 (+\mathbf{80_h})\)
  \item \(U_0 = B_0 = \mathbf{10}P_5 \dots P_0 (+\mathbf{80_h})\)
  \end{itemize}

\item 4 bytes
  \begin{itemize}
  \item \(U_3 = B_3 = \mathbf{11110}P_20 \dots P_18 (+\mathbf{F0_h})\)
  \item \(U_2 = B_2 = \mathbf{10}P_17 \dots P_12 (+\mathbf{80_h})\)
  \item \(U_1 = B_1 = \mathbf{10}P_11 \dots P_6 (+\mathbf{80_h})\)
  \item \(U_0 = B_0 = \mathbf{10}P_5 \dots P_0 (+\mathbf{80_h})\)
  \end{itemize}
\end{itemize}

\subsection*{UTF-16}
\textbf{Calculating UTF-16}:
\begin{itemize}
\item CP in [0; FFFF], 2 bytes / 1 CU (without D800\textsubscript{h} - DFFF\textsubscript{h})
  \begin{enumerate}
  \item U\textsubscript{0} = P\textsubscript{15} \ldots{} P\textsubscript{0}
  \end{enumerate}

\item CP in [1'0000\textsubscript{h}; 10'FFFF\textsubscript{h}], 4 bytes / 2 CU
  \begin{enumerate}
  \item Q = P - 1'0000\textsubscript{h} (Q\textsubscript{15} \ldots{} Q\textsubscript{0} = P\textsubscript{15} \ldots{} P\textsubscript{0})
  \item U\textsubscript{1} = 1101'10Q\textsubscript{19} \ldots{} Q\textsubscript{10} (+ \textbf{D800\textsubscript{h}});
  \item U\textsubscript{0} = 1101'11Q\textsubscript{9} \ldots{} Q0 (+ \textbf{DC00\textsubscript{h}})
  \end{enumerate}
\end{itemize}

\section*{ext 2 / ext 4}
\subsection*{direct / indirect block addressing}
\begin{enumerate}
\item Calculate numbers of needed blocks
  \begin{enumerate}
  \item FileSize / BlockSize = total neede Blocks
  \end{enumerate}
\item Calculate consecutive blocks
  \begin{enumerate}
  \item consecutiveSize / BlockSize = number of consectuive Blocks
  \end{enumerate}
\item Calculate how many consectuive blocks
  \begin{enumerate}
  \item total need blocks / number of consectuive blocks
  \end{enumerate}
\item Fill the array from 0 - B (11). Set C to empty meta data block
\item calculate last offset per consecutive blocks
  \begin{enumerate}
  \item number of consectuive blocks - D\textsubscript{h} (20\textsubscript{h} - D\textsubscript{h} = 13\textsubscript{h})
  \end{enumerate}
\item calculate last offset of whole data
  \begin{enumerate}
  \item total needed blocks - D\textsubscript{h} (400\textsubscript{h} - D\textsubscript{h} = 3F3\textsubscript{h})
  \end{enumerate}
\end{enumerate}

\subsection*{extend tree}
\begin{itemize}
\item first 12 bytes are the Extend Tree Header
\item Index nodes is a block which contains Extends
  \begin{itemize}
  \item 6 bytes for physical block numbers
  \item 4 bytes for the smallest logical block number
  \item contains a header after that the extends
  \item are more than \(4 \cdot 340 = 1360\) extends needed you can store additional index nodes instead of extends.
  \end{itemize}
\end{itemize}
\subsection*{journaling}
\begin{itemize}
\item Full Journal (everything in the journal, data and metadata)
\item Ordered (meta data in journal, content to disk, commit)
\item Writeback (meta data in journal, commit and write to disk in arbitary order)
\end{itemize}

\end{multicols}
\end{document}
